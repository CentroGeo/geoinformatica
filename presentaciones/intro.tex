
%%% Local Variables:
%%% mode: latex
%%% TeX-master: t
%%% End:
\documentclass[10pt]{beamer}
\usetheme[
%%% option passed to the outer theme
%    progressstyle=fixedCircCnt,   % fixedCircCnt, movingCircCnt (moving is deault)
  ]{Feather}
  
% If you want to change the colors of the various elements in the theme, edit and uncomment the following lines

% Change the bar colors:
%\setbeamercolor{Feather}{fg=red!20,bg=red}

% Change the color of the structural elements:
%\setbeamercolor{structure}{fg=red}

% Change the frame title text color:
%\setbeamercolor{frametitle}{fg=blue}

% Change the normal text color background:
%\setbeamercolor{normal text}{fg=black,bg=gray!10}

%-------------------------------------------------------
% INCLUDE PACKAGES
%-------------------------------------------------------

\usepackage[utf8]{inputenc}
\usepackage[english]{babel}
\usepackage[T1]{fontenc}
\usepackage{helvet}

%-------------------------------------------------------
% DEFFINING AND REDEFINING COMMANDS
%-------------------------------------------------------

% colored hyperlinks
\newcommand{\chref}[2]{
  \href{#1}{{\usebeamercolor[bg]{Feather}#2}}
}

%-------------------------------------------------------
% INFORMATION IN THE TITLE PAGE
%-------------------------------------------------------

\title[] % [] is optional - is placed on the bottom of the sidebar on every slide
{ % is placed on the title page
      \textbf{Taller de Geoinformática}
}

\subtitle[The Feather Beamer Theme]
{
      \textbf{(Geoinformática II)}
}

\author[Pablo López Ramírez]
{      Pablo López Ramírez \\
      {\ttfamily pablo.lopez@centrogeo.edu.mx}
}

\institute[]
{
      CentroGeo\\
      
  
  %there must be an empty line above this line - otherwise some unwanted space is added between the university and the country (I do not know why;( )
}

\date{\today}

%-------------------------------------------------------
% THE BODY OF THE PRESENTATION
%-------------------------------------------------------

\begin{document}

%-------------------------------------------------------
% THE TITLEPAGE
%-------------------------------------------------------

{\1% % this is the name of the PDF file for the background
\begin{frame}[plain,noframenumbering] % the plain option removes the header from the title page, noframenumbering removes the numbering of this frame only
  \titlepage % call the title page information from above
\end{frame}}


\begin{frame}{Contenido}{}
\tableofcontents
\end{frame}

%-------------------------------------------------------
\section{Introducción}
%-------------------------------------------------------

\begin{frame}{Introducción}
%-------------------------------------------------------
  El objetivo del curso es exponerlos a algunos temas más avanzados en Geoinformática.
  No se espera que al terminar el curso todos sean capaces de desarrollar soluciones
  sofisticadas, lo que se espera es que tengan un panorama general del desarrollo de
  aplicaciones para manejar, analizar y visualizar información geográfica.
  
\end{frame}

\begin{frame}{Introducción}
%-------------------------------------------------------
  En general, vamos a trabajar en Python, pero haremos algunas cosas en otros lenguajes
  dependiendo de las necesidades.

  Python, además de ser relativamente fácil de aprender, es el lenguaje en el que se
  extienden algunas de las herramientas más populares (ESRI, QGis)
\end{frame}


\section{Contenido del curso}
\begin{frame}{Contenido}
%-------------------------------------------------------

  \begin{itemize}
  \item Automatización
  \item Extendiendo Arc: ArcPy
  \item Desarrollo de Plugins para QGis
  \item Modelos 3-D en Qgis
  \item Dependiendo del tiempo y sus intereses
  \end{itemize}

\end{frame}


\end{document}
