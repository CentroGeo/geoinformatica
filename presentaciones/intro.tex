
%%% Local Variables:
%%% mode: latex
%%% TeX-master: t
%%% End:
\documentclass[10pt]{beamer}
\usetheme[
%%% option passed to the outer theme
%    progressstyle=fixedCircCnt,   % fixedCircCnt, movingCircCnt (moving is deault)
  ]{Feather}
  
% If you want to change the colors of the various elements in the theme, edit and uncomment the following lines

% Change the bar colors:
%\setbeamercolor{Feather}{fg=red!20,bg=red}

% Change the color of the structural elements:
%\setbeamercolor{structure}{fg=red}

% Change the frame title text color:
%\setbeamercolor{frametitle}{fg=blue}

% Change the normal text color background:
%\setbeamercolor{normal text}{fg=black,bg=gray!10}

%-------------------------------------------------------
% INCLUDE PACKAGES
%-------------------------------------------------------

\usepackage[utf8]{inputenc}
\usepackage[english]{babel}
\usepackage[T1]{fontenc}
\usepackage{helvet}

%-------------------------------------------------------
% DEFFINING AND REDEFINING COMMANDS
%-------------------------------------------------------

% colored hyperlinks
\newcommand{\chref}[2]{
  \href{#1}{{\usebeamercolor[bg]{Feather}#2}}
}

%-------------------------------------------------------
% INFORMATION IN THE TITLE PAGE
%-------------------------------------------------------

\title[] % [] is optional - is placed on the bottom of the sidebar on every slide
{ % is placed on the title page
      \textbf{Taller de Geoinformática}
}

\subtitle[Geoinformática]
{
      \textbf{(Geoinformática II)}
}

\setbeamerfont{author}{size=\scriptsize,series=\bfseries,parent=structure}
\author[JG,PLR,RTM]{Josafat Guerero \and Pablo López Ramírez
                    \and Rodrigo Tapia McClung}

\institute[CentroGeo]{\inst{1} CentroGeo}

% \author[Pablo López Ramírez]
% {      Josafat Guerero
%        Pablo López Ramírez
%        Rodrigo Tapia McClung \\
       
%        {\ttfamily pablo.lopez@centrogeo.edu.mx}
%        {\ttfamily josafatisai@gmail.com}
%        {\ttfamily rtapia@centrogeo.edu.mx}
% }

\institute[]
{
      CentroGeo\\
      
  
  %there must be an empty line above this line - otherwise some unwanted space is added between the university and the country (I do not know why;( )
}

\date{\today}

%-------------------------------------------------------
% THE BODY OF THE PRESENTATION
%-------------------------------------------------------

\begin{document}

%-------------------------------------------------------
% THE TITLEPAGE
%-------------------------------------------------------


\begin{frame}[plain,noframenumbering] % the plain option removes the header from the title page, noframenumbering removes the numbering of this frame only
  \titlepage % call the title page information from above
\end{frame}


\begin{frame}{Contenido}{}
\tableofcontents
\end{frame}

%-------------------------------------------------------
\section{Introducción}
%-------------------------------------------------------

\begin{frame}{Introducción}
  % -------------------------------------------------------
  \begin{block}{}
    El objetivo del curso es exponerlos a algunos temas más avanzados
    en Geoinformática.
    
    No se espera que al terminar el curso todos sean capaces de desarrollar soluciones
    sofisticadas, lo que se espera es que tengan un panorama general del desarrollo de
    aplicaciones para manejar, analizar y visualizar información geográfica.
  \end{block}
\end{frame}

\begin{frame}{Introducción}
%-------------------------------------------------------
  \begin{block}{}
    Vamos a trabajar con tres herramientas: Python, Javascript y R.

    Durante el curso haremos algunas cosas que ya han visto en otras materias, pero
    aquí el foco está en usar diferentes tecnologías para alcanzar los mismos
    objetivos.

  \end{block}
\end{frame}

%-------------------------------------------------------
\section{Contenido del curso}
% -------------------------------------------------------

\begin{frame}{Contenido}
%-------------------------------------------------------

  \begin{itemize}
  \item \textbf{Python:} Cálculos en vecindades 
  \item \textbf{QGis:} Modelos tridimensionales
  \item \textbf{R:} Estadística espacial
  \item \textbf{Javascript:} Mapas y gráficas en D3.js
  \item \textbf{Python:} ArcPy, extendiendo ArcMap
  \end{itemize}

\end{frame}


%-------------------------------------------------------
\section{Evaluación}
% -------------------------------------------------------

\begin{frame}{Evaluación}
%-------------------------------------------------------

  \begin{itemize}
  \item Tareas (en equipos de 3 personas) 20%
  \item Examen final (individual) 30%
  \item Proyecto final (en equipos de 3 personas) 50%
  \end{itemize}\pause

  \begin{block}{}
    Proyecto final:

    adquirir datos -> transformarlos -> analizarlos -> representarlos
  \end{block}

\end{frame}



\end{document}
