
%%% Local Variables:
%%% mode: latex
%%% TeX-master: t
%%% End:
\documentclass[10pt]{beamer}
\usetheme[
%%% option passed to the outer theme
%    progressstyle=fixedCircCnt,   % fixedCircCnt, movingCircCnt (moving is deault)
  ]{Feather}
  
% If you want to change the colors of the various elements in the theme, edit and uncomment the following lines

% Change the bar colors:
%\setbeamercolor{Feather}{fg=red!20,bg=red}

% Change the color of the structural elements:
%\setbeamercolor{structure}{fg=red}

% Change the frame title text color:
%\setbeamercolor{frametitle}{fg=blue}

% Change the normal text color background:
%\setbeamercolor{normal text}{fg=black,bg=gray!10}

%-------------------------------------------------------
% INCLUDE PACKAGES
%-------------------------------------------------------

\usepackage[utf8]{inputenc}
\usepackage[english]{babel}
\usepackage[T1]{fontenc}
\usepackage{helvet}

%-------------------------------------------------------
% DEFFINING AND REDEFINING COMMANDS
%-------------------------------------------------------

% colored hyperlinks
\newcommand{\chref}[2]{
  \href{#1}{{\usebeamercolor[bg]{Feather}#2}}
}

%-------------------------------------------------------
% INFORMATION IN THE TITLE PAGE
%-------------------------------------------------------

\title[] % [] is optional - is placed on the bottom of the sidebar on every slide
{ % is placed on the title page
      \textbf{Python}
}

\subtitle[Geoinformática]
{
      \textbf{Configurando el entorno}
}

\setbeamerfont{author}{size=\scriptsize,series=\bfseries,parent=structure}
\author[JG,PLR,RTM]{Josafat Guerero \and Pablo López Ramírez
                    \and Rodrigo Tapia McClung}

\institute[CentroGeo]{\inst{1} CentroGeo}

\institute[]
{
      CentroGeo\\
      
  
  %there must be an empty line above this line - otherwise some unwanted space is added between the university and the country (I do not know why;( )
}

\date{\today}

%-------------------------------------------------------
% THE BODY OF THE PRESENTATION
%-------------------------------------------------------

\begin{document}

%-------------------------------------------------------
% THE TITLEPAGE
%-------------------------------------------------------


\begin{frame}[plain,noframenumbering] % the plain option removes the header from the title page, noframenumbering removes the numbering of this frame only
  \titlepage % call the title page information from above
\end{frame}


\begin{frame}{Contenido}{}
\tableofcontents
\end{frame}

%-------------------------------------------------------
\section{Introducción}
%-------------------------------------------------------

\begin{frame}{Introducción}
  % -------------------------------------------------------
  \begin{block}{}
    Para estos talleres vamos a usar una distribución de Python llamada
    \textbf{Anaconda}
  \end{block}\pause
  
  \begin{block}{}
    \textbf{Anaconda} hace mucho más fácil el proceso de instalar paquetes y sus
    dependencias. Se encarga también de instalar los paquetes \textit{binarios}
    necesarios (¿recuerdan GDAL?)
  \end{block}

\end{frame}


%-------------------------------------------------------
\section{Conda}
% -------------------------------------------------------

\begin{frame}{Conda}
%-------------------------------------------------------
  \begin{block}{}
    Conda es el manejador de paquetes de \textbf{Anaconda}
  \end{block}\pause
  
  \begin{block}{}
    Es análogo a \textit{pip} pero además de manejar las dependencias de Python,
    también se encarga de dependencias \textit{binarias}, lo que facilita el
    proceso de instalación de librerías.
  \end{block}
\end{frame}


%-------------------------------------------------------
\section{Instalación}
% -------------------------------------------------------

\begin{frame}{Instalación}
%-------------------------------------------------------

  \begin{block}{}
    \textbf{Anaconda} contiene una distribución completa de Python científico,
    para ahorrar tiempo en la instalación, vamos a utilizar la distribución
    mínima: \textbf{Miniconda}
  \end{block}\pause

  \begin{block}{}
    Para instalar miniconda, visiten la página \chref{https://conda.io/miniconda.html}{https://conda.io/miniconda.html} y seleccionen la plataforma que requieran (windoze 64).

    Hagan doble click en el instalador y sigan las instrucciones
  \end{block}

\end{frame}



\end{document}
